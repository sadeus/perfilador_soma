    En el Laboratorio de Electrónica Cuántica (LEC) se dispone de un microscopio SPIM (de Single Plane Ilumination Microscope, o microscopio de iluminación de plano único). Este microscopio hace uso de muestras con fluoroforos e ilumina en cada instante las muestras con una hoja de laz (lightsheet). Esto elimina el photobleaching, o blanqueo de los flouroforos, y a su vez si se miden varias planos se puede generar una imagen 3D de la muestra.
        
    La construcción de este microscopio requiere determinar
    \begin{itemize}
        \item El perfil del haz a la entrada del microscopio, ya que este haz determina el tamaño de la hoja del haz.
        \item El perfil del haz a la salida del telescopio, antes del objetivo, ya que este debe enfocar en el objetivo para disminuir el tamaño del lightsheet
        \item El espectro del haz, ya que determina los flouroforos a utilizar
        \item Finalmente, la polarización del haz. Si esta polarización es lineal permite hacer mediciones de anisotropía, es decir la distribución en el espacio, de los fluoroforos
    \end{itemize}
            
    Con miras de poder calibrar este microscopio, se diseño y construyó instrumental portátil, que sea de fácil colocación y uso. En particular se construyeron
    \begin{itemize}
        \item Perfilador
        \item Polarimetro
    \end{itemize}



