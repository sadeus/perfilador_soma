\subsubsection{Mediciones con el perfilador}

        Calibración del instrumento, medición del colimador F280FC del SPIM. Se midió al inicio y al final del riel, para observar divergencias
        \begin{itemize}
                \item Inicio del riel:\\ $\sigma = (3,03 \pm 0,15)\,\text{mm}$
                \item Fin del riel:\\ $\sigma = (3,02 \pm 0,18)\,\text{mm}$
        \end{itemize}
        Colimador \underline{efectivamente colima el haz}.
        
        Luego se midicó de forma manual el haz, por medio de una hoja filosa y un tornillo microméticro, y se obtuvo lo que se ve en la figura. 
        
        De esta forma se puede aceverar que el perfilador tiene la misma precisión que el método manual, de alrededor de un 5\%, y además el sistema es exacto en su medición.
            
       
       Finalmente se hizo una medición extra en el telescopio del SPIM para poder observar la divergencia del haz. En la figura se puede observar una diferencia apreciable entre de diferente plano de obturación, lo que demuestra que a divergencia del haz es medible. 
       
       Nos queda poder utilizar el perfilador para determinar las propiedades de la lente cilindrica del SPIM.
 

\subsection{Mediciones con el polarimetro}
        Para poder medir con el polarimetro, primero es necesario calibrar el materia polarizador. Para eso se construyó dos láminas rotantes y se midió la intensidad de un haz linealmente polarizado atravezando ambas láminas alineadas (en el máximo y minimo de intensidad).
         Se encontró que la matriz de transmisión es la siguiente
            \begin{equation*}
                \begin{pmatrix} 0,5 & 0 \\ 0 & 2\times 10^{-6} \end{pmatrix}
            \end{equation*}
            hecho que marca al material polarizador lejos del ideal, ya que elimina el 50\% de la señal en el máximo y no elimina toda la polarización del haz, pero suficiente para hacer una medición cualitativa de la polarización. Para mejorar el sistema se debe utilizar un polarizador con mejores parámetros.
            
            
        Finalmente, se midió la polarización del laser más utilizado en el SPIM, modelo DHOM-M-473-150mW (que es azul en $\lambda = 473$nm), sacando un reflejo del haz antes de la fibra y el haz después de la fibra. Además se agrega el resultado del cociente para determinar la \emph{calidad de polarización lineal}. 
        
        Como se ve la polarización es casi lineal en ambos casos, pero lo más importante es que polarización se mantienen. Este análisis se hizo para varias potencias del haz, pero lamentablemente el polarizador deja de funcionar correctamente. 
        
        El setup también dispone de un láser rojo y un laser verde. El laser rojo se encontró que no mantiene polarización, ya que antes de la fibra el cociente es de $(0,94pm0,32)$ y $(0,39\pm 0,15)$, pero se volverán a medir, pero el laser verde tiene fluctuaciones de potencia por lo que no es muy útil para esta medición.
        
