
En laboratorio 7, se repensó el perfilador con un soporte con mayor agarre del motor, además de un sistema más simple para colocarlo rápidamente en la mesa óptica. El primer diseño se puede ver en la figura y tiene las siguiente propiedades
\begin{itemize}
    \item Soporte adosable a la mesa óptica por perros. Facil colocación
    \item Motor encastrado en soporte. No hay artefactos mecánicos
    \item Tambor de metal con superficie no reflectante
    \item No permite medir fácilmente en el otro eje. Habrá otra iteración
\end{itemize}
    
Posteriormente, se utilizó un motor más pequeño, un NEMA 8, para achicar aún más el diseño (unas 5 veces). De esta forma este soporte permite utilizarse fácilmente en ambos ejes, y no solo eso, se encontró que el tambor de plástico funciona correctamente para perfilar la señal.

