\begin{frame}{Conclusiones}
    \begin{onlyenv}<1>
    Del polarizador
        \begin{itemize}
            \item El perfilador mide exitosamente el haz, con un error del \%5, comparable con la medición manual 
            \item Se pudo caracterizar el haz a salir de la fibra y en el telescopio correctamente
            \item El perfilador fue capaz de medir la divergencia \underline{con solo un set de mediciones}.
            \item Para medir la salida del telescopio habrá que diseñar un perfilador más compacto.
        \end{itemize}
    \end{onlyenv}
    
    \begin{onlyenv}<2>
        Del polarimetro
        \begin{itemize}
            \item La lámina polarizadora utilizada está lejos de ser un polarizador perfecto, pero es funcional a la aplicación
            \item Se midió la polarización antes de acoplar en fibra y después de acoplar en fibra y \underline{no se observó cambio de polarización}
            \item Esta medición es de importancia fundamental para medir anisotropía de flouroforos en el SPIM
        \end{itemize}
    \end{onlyenv}
\end{frame}
