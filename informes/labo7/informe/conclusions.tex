Respecto al perfilador, con el prototipo final se logró mediciones con el mismo error cometido, un 5\%, que el sistema manual, además se observó correlación entre la medición automática y la manual. Además se pudo conmensurar correctamente la divergencia del haz dentro de un telescopio.  

La velocidad de adquisición, determinada por la velocidad del motor y la velocidad de transferencia de datos, permite observar el perfil en tiempo real, es decir unas 24 veces por segundo. De esta forma es posible alinear y observar cambios del tamaño del haz al mismo tiempo, mejorando el proceso.

Además de mejorar la velocidad de adquisición de un perfil de haz, este instrumental permite medir sin desarmar el setup del SPIM, lo que permitiría ser utilizado fácilmente en cualquier experiencia del laboratorio y eventualmente adaptarlo para otras aplicaciones.

Mientras, la calibración de las láminas polarizadoras utilizadas determinó que distan de ser un polarizador perfecto. Sin embargo, no es necesario para descartar cambios en la polarización en el tiempo o polarización aleatoria. Estas láminas no permiten diferenciar un haz linealmente polarizado a uno elipticamente polarizado. 

Con las láminas calibradas se hicieron mediciones de la polaización antes y después de la fibra del SPIM, para varios laseres. Se observó que la polarización es definida y no cambia en los 2s de medición, lo que es un fuerte indicador de la constancia de dicha. Además la polarización de los laseres azul y verde se mantuvo al atravezar la fibra, pero no se observó el mismo resultado para el laser rojo. Es conveniente repetir estas mediciones con un polarizador más cercano al ideal.
