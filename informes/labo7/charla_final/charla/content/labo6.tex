 Dadas las motivaciones del trabajo, ahora se presenta una reseña del proyecto concluido en laboratorio 6
 
\subsection{Perfilador en Laboratorio 6}

El diseño del perfilador al finalizar laboratorio 6 consiste en el siguiente, que tiene como propiedades importantes
\begin{itemize}
    \item Diseño autoportante.
    \item Tambor de perfilación permite medir en sistema Cage de Thorlabs. Permite medir divergencias.
    \item Tambor impreso en 3D.
    \item Motor paso a paso NEMA 17. 200 pasos por vuelta. Máximo 15rps. 
\end{itemize}
            

\subsection{Electrónica de adquisición}

\begin{itemize}
    \item uC Teensy v3.2. CPU 96MHz, y 64KiB RAM. ADC 1Msps max.
    \item Pololu A4988. Motores hasta 1.5A por fase
    \item Buffer de puerto serie de 1200 datos
    \item Software de ajuste en continuo cambio. Hecho en Python
    \item Máxima adquisición de 12 perfiles por segundo, limitación del uC/Software.
\end{itemize}
        

\subsection{Mediciones de calibración}
        Finalmente, se hicieron mediciones de calibración, a la entrada del SPIM, que en ese momento usaba un colimador F220. Se puede observar una diferencia importante en la perfilación entre distintas obturaciones; en laboratorio 7 se pudo determinar que el origen de esta diferencia es mecánica, por lo que se hicieron mejoras en los diseños
         
        Si se hace zoom sobre esta medición podemos ver que la función error no ajusta correctamente al empezar a obturar el haz. Esto se considera que es un problema de adquisición, debido a la resistencia de carga utilizada. Se considera amplificar la señal del fotodiodo
           
